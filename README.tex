\documentclass[]{article}
\usepackage[T1]{fontenc}
\usepackage{lmodern}
\usepackage{amssymb,amsmath}
\usepackage{ifxetex,ifluatex}
\usepackage{fixltx2e} % provides \textsubscript
% Set line spacing
% use upquote if available, for straight quotes in verbatim environments
\IfFileExists{upquote.sty}{\usepackage{upquote}}{}
\ifnum 0\ifxetex 1\fi\ifluatex 1\fi=0 % if pdftex
  \usepackage[utf8]{inputenc}
\else % if luatex or xelatex
  \ifxetex
    \usepackage{mathspec}
    \usepackage{xltxtra,xunicode}
  \else
    \usepackage{fontspec}
  \fi
  \defaultfontfeatures{Mapping=tex-text,Scale=MatchLowercase}
  \newcommand{\euro}{€}
\fi
% use microtype if available
\IfFileExists{microtype.sty}{\usepackage{microtype}}{}
\usepackage[margin=1in]{geometry}
\ifxetex
  \usepackage[setpagesize=false, % page size defined by xetex
              unicode=false, % unicode breaks when used with xetex
              xetex]{hyperref}
\else
  \usepackage[unicode=true]{hyperref}
\fi
\hypersetup{breaklinks=true,
            bookmarks=true,
            pdfauthor={},
            pdftitle={},
            colorlinks=true,
            citecolor=blue,
            urlcolor=blue,
            linkcolor=magenta,
            pdfborder={0 0 0}}
\urlstyle{same}  % don't use monospace font for urls
\setlength{\parindent}{0pt}
\setlength{\parskip}{6pt plus 2pt minus 1pt}
\setlength{\emergencystretch}{3em}  % prevent overfull lines
\setcounter{secnumdepth}{0}

%%% Change title format to be more compact
\usepackage{titling}
\setlength{\droptitle}{-2em}
  \title{}
  \pretitle{\vspace{\droptitle}}
  \posttitle{}
  \author{}
  \preauthor{}\postauthor{}
  \date{}
  \predate{}\postdate{}




\begin{document}

\maketitle


\subsubsection{Introduction}\label{introduction}

This second programming assignment will require you to write an R
function that is able to cache potentially time-consuming computations.
For example, taking the mean of a numeric vector is typically a fast
operation. However, for a very long vector, it may take too long to
compute the mean, especially if it has to be computed repeatedly
(e.g.~in a loop). If the contents of a vector are not changing, it may
make sense to cache the value of the mean so that when we need it again,
it can be looked up in the cache rather than recomputed. In this
Programming Assignment you will take advantage of the scoping rules of
the R language and how they can be manipulated to preserve state inside
of an R object.

\subsubsection{Example: Caching the Mean of a
Vector}\label{example-caching-the-mean-of-a-vector}

In this example we introduce the \texttt{\textless{}\textless{}-}
operator which can be used to assign a value to an object in an
environment that is different from the current environment. Below are
two functions that are used to create a special object that stores a
numeric vector and caches its mean.

The first function, \texttt{makeVector} creates a special ``vector'',
which is really a list containing a function to

\begin{enumerate}
\def\labelenumi{\arabic{enumi}.}
\itemsep1pt\parskip0pt\parsep0pt
\item
  set the value of the vector
\item
  get the value of the vector
\item
  set the value of the mean
\item
  get the value of the mean
\end{enumerate}

\begin{verbatim}
makeVector <- function(x = numeric()) {
        m <- NULL
        set <- function(y) {
                x <<- y
                m <<- NULL
        }
        get <- function() x
        setmean <- function(mean) m <<- mean
        getmean <- function() m
        list(set = set, get = get,
             setmean = setmean,
             getmean = getmean)
}
\end{verbatim}

The following function calculates the mean of the special ``vector''
created with the above function. However, it first checks to see if the
mean has already been calculated. If so, it \texttt{get}s the mean from
the cache and skips the computation. Otherwise, it calculates the mean
of the data and sets the value of the mean in the cache via the
\texttt{setmean} function.

\begin{verbatim}
cachemean <- function(x, ...) {
        m <- x$getmean()
        if(!is.null(m)) {
                message("getting cached data")
                return(m)
        }
        data <- x$get()
        m <- mean(data, ...)
        x$setmean(m)
        m
}
\end{verbatim}

\subsubsection{Assignment: Caching the Inverse of a
Matrix}\label{assignment-caching-the-inverse-of-a-matrix}

Matrix inversion is usually a costly computation and there may be some
benefit to caching the inverse of a matrix rather than computing it
repeatedly (there are also alternatives to matrix inversion that we will
not discuss here). Your assignment is to write a pair of functions that
cache the inverse of a matrix.

Write the following functions:

\begin{enumerate}
\def\labelenumi{\arabic{enumi}.}
\itemsep1pt\parskip0pt\parsep0pt
\item
  \texttt{makeCacheMatrix}: This function creates a special ``matrix''
  object that can cache its inverse.
\item
  \texttt{cacheSolve}: This function computes the inverse of the special
  ``matrix'' returned by \texttt{makeCacheMatrix} above. If the inverse
  has already been calculated (and the matrix has not changed), then
  \texttt{cacheSolve} should retrieve the inverse from the cache.
\end{enumerate}

Computing the inverse of a square matrix can be done with the
\texttt{solve} function in R. For example, if \texttt{X} is a square
invertible matrix, then \texttt{solve(X)} returns its inverse.

For this assignment, assume that the matrix supplied is always
invertible.

In order to complete this assignment, you must do the following:

\begin{enumerate}
\def\labelenumi{\arabic{enumi}.}
\itemsep1pt\parskip0pt\parsep0pt
\item
  Fork the GitHub repository containing the stub R files at
  \href{https://github.com/rdpeng/ProgrammingAssignment2}{\url{https://github.com/rdpeng/ProgrammingAssignment2}}
  to create a copy under your own account.
\item
  Clone your forked GitHub repository to your computer so that you can
  edit the files locally on your own machine.
\item
  Edit the R file contained in the git repository and place your
  solution in that file (please do not rename the file).
\item
  Commit your completed R file into YOUR git repository and push your
  git branch to the GitHub repository under your account.
\item
  Submit to Coursera the URL to your GitHub repository that contains the
  completed R code for the assignment.
\end{enumerate}

\subsubsection{Grading}\label{grading}

This assignment will be graded via peer assessment.

\end{document}
